\documentclass{article}

% New commands declaration

\usepackage[frenchb]{babel}
\usepackage[T1]{fontenc}

\usepackage{natbib,bibentry}
\usepackage{color}
\usepackage{yfonts}
\usepackage{graphicx}
\usepackage{epsfig,subfigure}
\usepackage{amsmath,amssymb,amsfonts}
\usepackage{calc}

\usepackage{array}
\newcolumntype{L}[1]{>{\raggedright\let\newline\\\arraybackslash\hspace{0pt}}m{#1}}
\newcolumntype{C}[1]{>{\centering\let\newline\\\arraybackslash\hspace{0pt}}m{#1}}
\newcolumntype{R}[1]{>{\raggedleft\let\newline\\\arraybackslash\hspace{0pt}}m{#1}}


\DeclareGraphicsExtensions{.eps, .jpg, .png}

\parindent = 0mm

\bibliographystyle{plain}

\hoffset = -20mm
\voffset = -25mm
\textwidth = 160mm
\textheight = 240mm

% \definecolor{lightgray}{gray}{0.2}


\newcommand{\expect}{{\rm I \mkern-2.5mu \nonscript\mkern-.5mu E}}
\newcommand{\equaldef}{\stackrel{d}{=}}
\newcommand{\argmax}{\operatornamewithlimits{argmax}}

\newcommand{\dnu}{16}
\newcommand{\solskip}{10mm}

\renewcommand{\topfraction}{1}
\renewcommand{\bottomfraction}{1}
\renewcommand{\textfraction}{0}

\newcommand{\debutrep}[1]{\color{blue}\begin{center} \hrulefill \textbf{ #1 } \hrulefill \end{center} }
\newcommand{\finrep}{\vspace*{5mm}\hfill $\square$\color{black}\vspace*{5mm}}


\begin{document}

\baselineskip = 4mm
\title{Traitement des Signaux Aléatoires \\
Estimation Spectrale}
\author{\textbf{4 ETI -- CPE Lyon }\\[3mm]
{Travaux Pratiques TSA}}
\date{2020-2021}

\maketitle

\noindent\fbox{
\parbox{\linewidth-2\fboxrule-2\fboxsep}
{ 
\vspace*{2mm}
{\large\bf Noms, Prénoms: }\\[3mm]
{\large\bf Groupe: }\\[3mm]
{\large\bf Date:}\\[2mm]}}
\vspace*{5mm}


\textbf{\Large Objectifs du TP}\\[4mm]

\begin{list}{-}{\setlength{\leftmargin}{3mm} \setlength{\labelwidth}{20mm} \setlength{\labelsep}{2mm} \setlength{\itemsep}{1mm} }
\item Comprendre la notion de densité spectrale d'énergie ou de puissance moyenne
\item Manipuler différents estimateurs empiriques (à partir d'une série temporelle de taille finie) de DSE/DSPM
\item Etudier l'effet du compromis biais-variance d'un estimateur
\end{list}


\vspace*{5mm}

\section{Préparation}

\begin{itemize}
\item[{\bf Question 1}] Comment peut-on calculer simplement la densité spectrale d’énergie (DSE) d’un signal certain d’énergie finie ?

\debutrep{réponse}

\finrep

\item[{\bf Question 2}] Comment est définie la densité spectrale de puissance moyenne (DSPM) d’un processus aléatoire ?

\debutrep{réponse}

\finrep

\item[{\bf Question 3}] Quelles sont les grandeurs qui permettent de chiffrer la qualité d’une estimation dans le cas général ? et la qualité de l’estimation spectrale en particulier.

\debutrep{réponse}

\finrep

\item[{\bf Question 4}] Exprimer la densité spectrale de puissance moyenne (DSPM) GB ( f ) d’un bruit blanc stationnaire centré.

\debutrep{réponse}

\finrep

\item[{\bf Question 5}] Exprimer GX ( f ) , où X(t) est la sortie d’un filtre excité par un bruit blanc centré, en fonction de la DSPM du bruit blanc et des caractéristiques du filtre.

\debutrep{réponse}

\finrep

\item[{\bf Question 6}] En une phrase (sans formule), décrire le procédé de calcul de la DSPM estimée G1 ( f )
d’une séquence aléatoire via l’estimateur simple.

\debutrep{réponse}

\finrep

\item[{\bf Question 7}] Rappeler le mode de graduation d’une TFD-N points en fréquences réduites.

\debutrep{réponse}

\finrep

\item[{\bf Question 8}] Décrire (avec une phrase) le procédé de calcul de la DSPM estimée G2 ( f ) d’une
séquence aléatoire via l’estimateur moyenné.

\debutrep{réponse}

\finrep

\item[{\bf Question 9}] Que signifie le terme «compromis biais-variance» dans le cas de l’estimateur moyenné ?

\debutrep{réponse}

\finrep

\item[{\bf Question 10}] Quelles modifications sont apportées au procédé de calcul de l’estimateur de Welch par rapport à l’estimateur moyenné ?

\debutrep{réponse}

\finrep

\end{itemize}

\clearpage
\setcounter{section}{2}
\section{Estimation de la DSPM d'un bruit blanc gaussien filtré}
\subsection{Génération du bruit à analyser}

A quoi sert l'entier permettant d'initialiser le générateur ?

\debutrep{réponse ci-dessous}

\finrep

\subsection{Estimateur spectral simple}
\subsubsection{Script de la fonction {\tt Matlab} développée}

\debutrep{code ci-dessous}
\begin{verbatim}

\end{verbatim}
\finrep

\subsubsection{Expérimentation}

\begin{enumerate}
\renewcommand{\theenumi}{\Alph{enumi}}
\item Etude du biais et de la variance en fonction du nombre $N$ d'échantillons de bruit

\debutrep{figures ci-dessous}

\begin{figure}[h]

\caption{$N$ faible -- indice de début de la séquence à 1}
\end{figure}

\begin{figure}[h]

\caption{$N$ élevé -- indice de début de la séquence à 1}
\end{figure}
\finrep

Commentaires.

\debutrep{réponse ci-dessous}

\finrep

\item Etude du biais et de la variance en fonction de la réalisation considérée, à $N$ fixé

\debutrep{figures ci-dessous}

\begin{figure}[h]

\caption{$N \sim 1000$ -- indice de début de la séquence à 1}
\end{figure}

\begin{figure}[h]

\caption{$N \sim 1000$  -- indice de début de la séquence fixé à une autre position ($\gg 1000$) dans la séquence}
\end{figure}
\finrep

Commentaires.

\debutrep{réponse ci-dessous}

\finrep

\item Etude du biais et de la variance en fonction du nombre $NFFT$ de FFT

\debutrep{figures ci-dessous}

\begin{figure}[h]

\caption{$N$ constant -- indice de début de la séquence à 1 -- $NFFT \sim N$}
\end{figure}

\begin{figure}[h]

\caption{$N$ constant -- indice de début de la séquence à 1 -- $NFFT \gg N$}
\end{figure}

\finrep

Commentaires.

\debutrep{réponse ci-dessous}

\finrep

\item Conclusion
Quel est le principal défaut de l'estimateur simple ?

\debutrep{réponse ci-dessous}

\finrep

\end{enumerate}

\clearpage
\subsection{Estimateur spectral moyenné}

\textbf{On fixera $\mathbf{N=4096}$ dans tout ce qui suit.}

\subsubsection{Script de la fonction {\tt Matlab} développé}

\debutrep{code ci-dessous}
\begin{verbatim}

\end{verbatim}
\finrep


\subsubsection{Expérimentation}

En précisant bien la valeur des paramètres utilisés pour les essais, affichez les figures correspondantes aux conditions indiquées.
\debutrep{figure ci-dessous}

\begin{figure}[h]

\caption{$N=4096$ -- tranches courtes $M = ??? $, $NFFT = ???$}
\end{figure}
\finrep

Commentaires
\debutrep{réponse ci-dessous}

\finrep

\debutrep{figure ci-dessous}
\begin{figure}[h]

\caption{$N=4096$ -- tranches longues $M = ???$, $NFFT = ???$}
\end{figure}

\finrep

Commentaires
\debutrep{réponse ci-dessous}

\finrep

\debutrep{figure ci-dessous}
\begin{figure}[h]

\caption{$N=4096$ -- \og Meilleur \fg compromis biais variance atteint pour $M = ???$, $NFFT = ???$}
\end{figure}

\finrep

Quelle information permettrait d'obtenir le meilleur compromis biais-variance? 

\debutrep{réponse ci-dessous}

\finrep

\clearpage
\section{Estimateur de Welch}

\subsection{Etude préalable des fenêtres}

Quelles différences de comportement fréquentiel peut-on observer pour les 6 fenêtres proposées (lobe principal, lobes latéraux\ldots).

\subsubsection{Script de la fonction {\tt Matlab} développée}

\debutrep{code ci-dessous}
\begin{verbatim}

\end{verbatim}
\finrep

\subsubsection{Expérimentation}

\begin{enumerate}
\renewcommand{\theenumi}{\Alph{enumi}}

\item Etude du biais et de la variance en fonction du taux de recouvrement entre tranches

Pour $N$, $M$ et $NFFT$ fixés et pour une  fenêtre choisie,  tracez les figures correspondantes aux conditions indiquées ci-dessous.

\debutrep{figure ci-dessous}
\begin{figure}[h]

\caption{$N = 4096$ -- $M = ???$, $NFFT = ???$. Choix de fenêtre = ??? -- Recouvrement de $0\%$}
\end{figure}

\begin{figure}[h]

\caption{$N = 4096$ -- $M = ???$, $NFFT = ???$. Choix de fenêtre = ??? -- Recouvrement de $50\%$}
\end{figure}

\finrep

Que permet le recouvrement entre tranches ?

\debutrep{réponse ci-dessous}

\finrep

\item Etude du biais et de la variance en fonction de la fenêtre utilisée

Pour $N$, $M$ et $NFFT$ fixés et pour différents choix de fenêtre,  tracez les figures correspondantes aux conditions indiquées ci-dessous.

\debutrep{figure ci-dessous}
\begin{figure}[h]

\caption{$N = 4096$ -- $M = ???$, $NFFT = ???$. Fenêtre Rectangle -- Recouvrement de $50\%$}
\end{figure}

\begin{figure}[h]

\caption{$N = 4096$ -- $M = ???$, $NFFT = ???$. Fenêtre ??? -- Recouvrement de $50\%$}
\end{figure}

\finrep

Que permet l'utilisation d'une fenêtre autre que rectangulaire ? Expliquer.

\debutrep{réponse ci-dessous}

\finrep

\end{enumerate}

Pour quelles valeurs des paramètres d'analyse obtenez vous le \og meilleur \fg résultat (celui qui vous parait le plus satisfaisant)?

\debutrep{réponse ci-dessous}

Longueur de la séquence analysée $N = ???$ \\
Longueur des tranches $M = ???$ \\
Type de fenêtre ??? \\
Taux de recouvrement = ??? \\
Nombre de points de transformée de Fourier $NFFT = ???$

\finrep

\section{Utilisation des estimateurs précédents pour analyser un signal inconnu}

\subsection{Modification des programmes}

Script d'une des fonctions modifiée

\debutrep{code ci-dessous}
\begin{verbatim}

\end{verbatim}
\finrep

\subsection{Expérimentation}

Afficher les spectres estimés obtenus avec chacune des 3 méthodes étudiées.

\debutrep{figure ci-dessous}
\begin{figure}[h]

\caption{Estimateur spectral simple. \textit{Indiquez tous les paramètres choisis}.}
\end{figure}

\begin{figure}[h]

\caption{Estimateur spectral moyenné. \textit{Indiquez tous les paramètres choisis}.}
\end{figure}

\begin{figure}[h]

\caption{Estimateur spectral de Welch. \textit{Indiquez tous les paramètres choisis}.}
\end{figure}

\finrep

Décrivez précisément la démarche expérimentale suivie. Avec quelle méthode êtes vous capable avec certitude de décrire le contenu fréquentiel de ce signal ?

\debutrep{réponse ci-dessous}

\finrep

\subsubsection{Interprétations}

\begin{enumerate}
\renewcommand{\theenumi}{\Alph{enumi}}

\item Que inconvénient majeur l'utilisation d'une fenêtre (d'apodisation en temps) engendre-t-elle?

\debutrep{réponse ci-dessous}

\finrep

\item Décrire (sans dessin) la forme de la DSPM obtenue.

\debutrep{réponse ci-dessous}

\finrep

\item Quelles informations la forme de cette DSPM apporte-t-elle sur le contenu (la nature) du signal?

\debutrep{réponse ci-dessous}

\finrep

\item Quelles mesures concernant  les caractéristiques du signal peut-on effectuer sur la DSPM?

\debutrep{réponse ci-dessous}

\finrep

\end{enumerate}

\end{document}








































